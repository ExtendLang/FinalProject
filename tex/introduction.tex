\chapter{Introduction}
Extend is a declarative programming language meant to support spreadsheet-like functionality. It contains features such as side-effect-free values, immutability, and automatic formula adjustments relative to rows and columns. Extend is compiled to the LLVM (Low Level Virtual Machine) intermediate representation, which in turn is reduced to machine assembly. Extend takes inspiration from software such as Microsoft Excel, which allows users to link several formulae on dependent groups of data together, but takes this technology a step further by allowing users to encapsulate such calculations as functions.

\section{Inspiration \& Use Cases}

	\subsection{Inspiration}
	The design goal of our language was to be "a spreadsheet you can compile". Extend was conceptualized to address the limitations that prevented the spreadsheet environment from evolving into a compiled, flexible programming language. To create this, there were three main things that needed to be changed about the way interactive spreadsheets work:
		\begin{itemize}
			\item The language needs reusable functions as opposed to having to copy \& paste a block of cells.
			\item Cell ranges need to be created with dynamic runtime-determined dimensions.
			\item Cells need to be able to contain composite values in addition to single numbers or strings.
		\end{itemize}
	With these changes in mind, we attempted to keep the semantics as similar as possible to traditional spreadsheet programs; this meant implementing a dynamically typed language that is tolerant of potential errors in its input data. Extend degrades gracefully in the presence of potential data errors.
	\newline \newline
	Spreadsheet applications cannot be 'run' on different sets of input data. Extend was conceptualized as a language to create standalone executables that can be repeatedly run on multiple files, theremby removing the need to manually enter inputs. In building this language, our mission was to bring the best of spreadsheets and computation into one product.

	\subsection{Complex Calculations Across Many Inputs}
	Extend is spiritually closer in behavior to Microsoft Excel than traditional imperative programming languages. The order of computation is determined implicitly by the language rather than explicitly by the developer. In addition, in one line of code, a single formula can be assigned to all the cells in a variable. The feature acts similarly to Python's list comprehension, or OCaml's \texttt{List.map} functionality.

	\subsection{Flexibility}
	Extend allows the dimensions of ranges to be determined dynamically at runtime, and handles most type errors by degrading gracefully instead of crashing the program. The standard library that Extend delivers includes a subset of the functions that are built into conventional spreadsheet applications. As many of these as possible were implemented in Extend itself.

\section{Target Audience}
Our target users are people who are proficient using spreadsheets and who are bumping up against the limits of what can be done with them, but who have only limited exposure to traditional imperative programming; perhaps a brief exposure to a high-level language such as Java, Python, or Javascript. This target audience dictated several of the design decisions we made about the language behavior:
\begin{itemize}
	\item A single Number type instead of separate types for integers and floating-point numbers
	\item Expressions involving incompatible types on the right-hand-side evaluate to \texttt{empty} instead of causing a runtime error
	\item Applicative instead of normal order for function calls
	\item Our selection operator automatically dereferences single-cell selections. We wanted sum(x[0:10,0:10]) to return the sum of a range, but x[0,0] + x[1,1] to return the sum of two numbers.
\end{itemize}
