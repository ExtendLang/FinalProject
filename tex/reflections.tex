\chapter{Reflection}

\section{Ishaan}
When working on a long-term project, communication is paramount. Throughout this project, I realized that maximum productivity occurred when the team kept a constant line of communication open regarding language and code design.
Additionally, it's important to identify where people can be most productive. If one person is more efficient at a certain task, more progress will be made if they work on similar material.
Lastly, as the design of the language evolves over time, it's important to build a system that allows for flexibility, as you never know what may change later in the development process.

\section{Jared}

\section{Nigel}
Team projects by its nature are a very unique challenge for a student. Nevertheless these projects are incredibly valuable by providing a more applied experience. Thus, I am glad that I was able to put a lot of effort into this project. Communication proved to be a key element in the project: We had weekly team meetings, meetings with the TA, a chatroom and ad hoc in person discussions. All this helped to bounce ideas off one another, prioritize well and avoid a mismatch in expectations.
Of course some problems are inevitable. Therefore I think one of our key assets was our test suite. At any point in time it allowed us to see the next step ahead - the next thing we want to make work. In the same vein our code review process proved very effective (PR required approval plus passing CI).
I admit that at some points in the development process I was slacking off, especially when facing LLVM codegen for the first time. However I am glad, that my team mates motivated me and helped me to get back on track.
Summarizing, every project has its issues, but by planning ahead and hard work, we built a surprisingly good and feature complete language that is close to our initial goal.

\section{Kevin}
